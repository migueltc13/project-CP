\documentclass[11pt, a4paper, fleqn]{article}
\usepackage{cp2425t}
\makeindex

%================= lhs2tex=====================================================%
%% ODER: format ==         = "\mathrel{==}"
%% ODER: format /=         = "\neq "
%
%
\makeatletter
\@ifundefined{lhs2tex.lhs2tex.sty.read}%
  {\@namedef{lhs2tex.lhs2tex.sty.read}{}%
   \newcommand\SkipToFmtEnd{}%
   \newcommand\EndFmtInput{}%
   \long\def\SkipToFmtEnd#1\EndFmtInput{}%
  }\SkipToFmtEnd

\newcommand\ReadOnlyOnce[1]{\@ifundefined{#1}{\@namedef{#1}{}}\SkipToFmtEnd}
\usepackage{amstext}
\usepackage{amssymb}
\usepackage{stmaryrd}
\DeclareFontFamily{OT1}{cmtex}{}
\DeclareFontShape{OT1}{cmtex}{m}{n}
  {<5><6><7><8>cmtex8
   <9>cmtex9
   <10><10.95><12><14.4><17.28><20.74><24.88>cmtex10}{}
\DeclareFontShape{OT1}{cmtex}{m}{it}
  {<-> ssub * cmtt/m/it}{}
\newcommand{\texfamily}{\fontfamily{cmtex}\selectfont}
\DeclareFontShape{OT1}{cmtt}{bx}{n}
  {<5><6><7><8>cmtt8
   <9>cmbtt9
   <10><10.95><12><14.4><17.28><20.74><24.88>cmbtt10}{}
\DeclareFontShape{OT1}{cmtex}{bx}{n}
  {<-> ssub * cmtt/bx/n}{}
\newcommand{\tex}[1]{\text{\texfamily#1}}	% NEU

\newcommand{\Sp}{\hskip.33334em\relax}


\newcommand{\Conid}[1]{\mathit{#1}}
\newcommand{\Varid}[1]{\mathit{#1}}
\newcommand{\anonymous}{\kern0.06em \vbox{\hrule\@width.5em}}
\newcommand{\plus}{\mathbin{+\!\!\!+}}
\newcommand{\bind}{\mathbin{>\!\!\!>\mkern-6.7mu=}}
\newcommand{\rbind}{\mathbin{=\mkern-6.7mu<\!\!\!<}}% suggested by Neil Mitchell
\newcommand{\sequ}{\mathbin{>\!\!\!>}}
\renewcommand{\leq}{\leqslant}
\renewcommand{\geq}{\geqslant}
\usepackage{polytable}

%mathindent has to be defined
\@ifundefined{mathindent}%
  {\newdimen\mathindent\mathindent\leftmargini}%
  {}%

\def\resethooks{%
  \global\let\SaveRestoreHook\empty
  \global\let\ColumnHook\empty}
\newcommand*{\savecolumns}[1][default]%
  {\g@addto@macro\SaveRestoreHook{\savecolumns[#1]}}
\newcommand*{\restorecolumns}[1][default]%
  {\g@addto@macro\SaveRestoreHook{\restorecolumns[#1]}}
\newcommand*{\aligncolumn}[2]%
  {\g@addto@macro\ColumnHook{\column{#1}{#2}}}

\resethooks

\newcommand{\onelinecommentchars}{\quad-{}- }
\newcommand{\commentbeginchars}{\enskip\{-}
\newcommand{\commentendchars}{-\}\enskip}

\newcommand{\visiblecomments}{%
  \let\onelinecomment=\onelinecommentchars
  \let\commentbegin=\commentbeginchars
  \let\commentend=\commentendchars}

\newcommand{\invisiblecomments}{%
  \let\onelinecomment=\empty
  \let\commentbegin=\empty
  \let\commentend=\empty}

\visiblecomments

\newlength{\blanklineskip}
\setlength{\blanklineskip}{0.66084ex}

\newcommand{\hsindent}[1]{\quad}% default is fixed indentation
\let\hspre\empty
\let\hspost\empty
\newcommand{\NB}{\textbf{NB}}
\newcommand{\Todo}[1]{$\langle$\textbf{To do:}~#1$\rangle$}

\EndFmtInput
\makeatother
%
%
%
%
%
%
% This package provides two environments suitable to take the place
% of hscode, called "plainhscode" and "arrayhscode". 
%
% The plain environment surrounds each code block by vertical space,
% and it uses \abovedisplayskip and \belowdisplayskip to get spacing
% similar to formulas. Note that if these dimensions are changed,
% the spacing around displayed math formulas changes as well.
% All code is indented using \leftskip.
%
% Changed 19.08.2004 to reflect changes in colorcode. Should work with
% CodeGroup.sty.
%
\ReadOnlyOnce{polycode.fmt}%
\makeatletter

\newcommand{\hsnewpar}[1]%
  {{\parskip=0pt\parindent=0pt\par\vskip #1\noindent}}

% can be used, for instance, to redefine the code size, by setting the
% command to \small or something alike
\newcommand{\hscodestyle}{}

% The command \sethscode can be used to switch the code formatting
% behaviour by mapping the hscode environment in the subst directive
% to a new LaTeX environment.

\newcommand{\sethscode}[1]%
  {\expandafter\let\expandafter\hscode\csname #1\endcsname
   \expandafter\let\expandafter\endhscode\csname end#1\endcsname}

% "compatibility" mode restores the non-polycode.fmt layout.

\newenvironment{compathscode}%
  {\par\noindent
   \advance\leftskip\mathindent
   \hscodestyle
   \let\\=\@normalcr
   \let\hspre\(\let\hspost\)%
   \pboxed}%
  {\endpboxed\)%
   \par\noindent
   \ignorespacesafterend}

\newcommand{\compaths}{\sethscode{compathscode}}

% "plain" mode is the proposed default.
% It should now work with \centering.
% This required some changes. The old version
% is still available for reference as oldplainhscode.

\newenvironment{plainhscode}%
  {\hsnewpar\abovedisplayskip
   \advance\leftskip\mathindent
   \hscodestyle
   \let\hspre\(\let\hspost\)%
   \pboxed}%
  {\endpboxed%
   \hsnewpar\belowdisplayskip
   \ignorespacesafterend}

\newenvironment{oldplainhscode}%
  {\hsnewpar\abovedisplayskip
   \advance\leftskip\mathindent
   \hscodestyle
   \let\\=\@normalcr
   \(\pboxed}%
  {\endpboxed\)%
   \hsnewpar\belowdisplayskip
   \ignorespacesafterend}

% Here, we make plainhscode the default environment.

\newcommand{\plainhs}{\sethscode{plainhscode}}
\newcommand{\oldplainhs}{\sethscode{oldplainhscode}}
\plainhs

% The arrayhscode is like plain, but makes use of polytable's
% parray environment which disallows page breaks in code blocks.

\newenvironment{arrayhscode}%
  {\hsnewpar\abovedisplayskip
   \advance\leftskip\mathindent
   \hscodestyle
   \let\\=\@normalcr
   \(\parray}%
  {\endparray\)%
   \hsnewpar\belowdisplayskip
   \ignorespacesafterend}

\newcommand{\arrayhs}{\sethscode{arrayhscode}}

% The mathhscode environment also makes use of polytable's parray 
% environment. It is supposed to be used only inside math mode 
% (I used it to typeset the type rules in my thesis).

\newenvironment{mathhscode}%
  {\parray}{\endparray}

\newcommand{\mathhs}{\sethscode{mathhscode}}

% texths is similar to mathhs, but works in text mode.

\newenvironment{texthscode}%
  {\(\parray}{\endparray\)}

\newcommand{\texths}{\sethscode{texthscode}}

% The framed environment places code in a framed box.

\def\codeframewidth{\arrayrulewidth}
\RequirePackage{calc}

\newenvironment{framedhscode}%
  {\parskip=\abovedisplayskip\par\noindent
   \hscodestyle
   \arrayrulewidth=\codeframewidth
   \tabular{@{}|p{\linewidth-2\arraycolsep-2\arrayrulewidth-2pt}|@{}}%
   \hline\framedhslinecorrect\\{-1.5ex}%
   \let\endoflinesave=\\
   \let\\=\@normalcr
   \(\pboxed}%
  {\endpboxed\)%
   \framedhslinecorrect\endoflinesave{.5ex}\hline
   \endtabular
   \parskip=\belowdisplayskip\par\noindent
   \ignorespacesafterend}

\newcommand{\framedhslinecorrect}[2]%
  {#1[#2]}

\newcommand{\framedhs}{\sethscode{framedhscode}}

% The inlinehscode environment is an experimental environment
% that can be used to typeset displayed code inline.

\newenvironment{inlinehscode}%
  {\(\def\column##1##2{}%
   \let\>\undefined\let\<\undefined\let\\\undefined
   \newcommand\>[1][]{}\newcommand\<[1][]{}\newcommand\\[1][]{}%
   \def\fromto##1##2##3{##3}%
   \def\nextline{}}{\) }%

\newcommand{\inlinehs}{\sethscode{inlinehscode}}

% The joincode environment is a separate environment that
% can be used to surround and thereby connect multiple code
% blocks.

\newenvironment{joincode}%
  {\let\orighscode=\hscode
   \let\origendhscode=\endhscode
   \def\endhscode{\def\hscode{\endgroup\def\@currenvir{hscode}\\}\begingroup}
   %\let\SaveRestoreHook=\empty
   %\let\ColumnHook=\empty
   %\let\resethooks=\empty
   \orighscode\def\hscode{\endgroup\def\@currenvir{hscode}}}%
  {\origendhscode
   \global\let\hscode=\orighscode
   \global\let\endhscode=\origendhscode}%

\makeatother
\EndFmtInput
%


%------------------------------------------------------------------------------%


%====== DEFINIR GRUPO E ELEMENTOS =============================================%

\group{G99}
\studentA{xxxxxx}{Nome }
\studentB{xxxxxx}{Nome }
\studentC{xxxxxx}{Nome }

%==============================================================================%

\begin{document}

\sffamily
\setlength{\parindent}{0em}
\emergencystretch 3em
\renewcommand{\baselinestretch}{1.25} 
\input{Cover}
\pagestyle{pagestyle}
\setlength{\parindent}{1em}
\newgeometry{left=25mm,right=20mm,top=25mm,bottom=25mm}

\section*{Preâmbulo}

Em \CP\ pretende-se ensinar a progra\-mação de computadores como uma disciplina
científica. Para isso parte-se de um repertório de \emph{combinadores} que
formam uma álgebra da programação % (conjunto de leis universais e seus corolários)
e usam-se esses combinadores para construir programas \emph{composicionalmente},
isto é, agregando programas já existentes.

Na sequência pedagógica dos planos de estudo dos cursos que têm esta disciplina,
opta-se pela aplicação deste método à programação em \Haskell\ (sem prejuízo
da sua aplicação a outras linguagens funcionais). Assim, o presente trabalho
prático coloca os alunos perante problemas concretos que deverão ser implementados
em \Haskell. Há ainda um outro objectivo: o de ensinar a documentar programas,
a validá-los e a produzir textos técnico-científicos de qualidade.

Antes de abordarem os problemas propostos no trabalho, os grupos devem ler
com atenção o anexo \ref{sec:documentacao} onde encontrarão as instruções
relativas ao \emph{software} a instalar, etc.

Valoriza-se a escrita de \emph{pouco} código que corresponda a soluções simples
e elegantes que utilizem os combinadores de ordem superior estudados na disciplina.


\Problema

Esta questão aborda um problema que é conhecido pela designação '\emph{H-index of a Histogram}'
e que se formula facilmente:
\begin{quote}\em
O h-index de um histograma é o maior número \ensuremath{\Varid{n}} de barras do histograma cuja altura é maior ou igual a \ensuremath{\Varid{n}}.
\end{quote}
Por exemplo, o histograma 
\begin{hscode}\SaveRestoreHook
\column{B}{@{}>{\hspre}l<{\hspost}@{}}%
\column{E}{@{}>{\hspre}l<{\hspost}@{}}%
\>[B]{}\Varid{h}\mathrel{=}[\mskip1.5mu \mathrm{5},\mathrm{2},\mathrm{7},\mathrm{1},\mathrm{8},\mathrm{6},\mathrm{4},\mathrm{9}\mskip1.5mu]{}\<[E]%
\ColumnHook
\end{hscode}\resethooks
que se mostra na figura
	\histograma
tem \ensuremath{\Varid{hindex}\;\Varid{h}\mathrel{=}\mathrm{5}}
pois há \ensuremath{\mathrm{6}} colunas maiores que \ensuremath{\mathrm{5}}. (Não é \ensuremath{\mathrm{6}} pois maiores ou iguais que seis só há quatro.)

Pretende-se definida como um catamorfismo, anamorfismo ou hilomorfismo uma função em Haskell
\begin{hscode}\SaveRestoreHook
\column{B}{@{}>{\hspre}l<{\hspost}@{}}%
\column{E}{@{}>{\hspre}l<{\hspost}@{}}%
\>[B]{}\Varid{hindex}\mathbin{::}[\mskip1.5mu \Conid{Int}\mskip1.5mu]\to (\Conid{Int},[\mskip1.5mu \Conid{Int}\mskip1.5mu]){}\<[E]%
\ColumnHook
\end{hscode}\resethooks
tal que, para \ensuremath{(\Varid{i},\Varid{x})\mathrel{=}\Varid{hindex}\;\Varid{h}}, \ensuremath{\Varid{i}} é o H-index de \ensuremath{\Varid{h}} e \ensuremath{\Varid{x}} é a lista de colunas de \ensuremath{\Varid{h}} que para ele contribuem.

A proposta de \ensuremath{\Varid{hindex}} deverá vir acompanhada de um \textbf{diagrama} ilustrativo.

\Problema

Pelo \href{https://en.wikipedia.org/wiki/Fundamental_theorem_of_arithmetic}{teorema
fundamental da aritmética}, todo número inteiro positivo tem uma única factorização
prima.  For exemplo,
\begin{tabbing}\ttfamily
~primes~455\\
\ttfamily ~\char91{}5\char44{}7\char44{}13\char93{}\\
\ttfamily ~primes~433\\
\ttfamily ~\char91{}433\char93{}\\
\ttfamily ~primes~230\\
\ttfamily ~\char91{}2\char44{}5\char44{}23\char93{}
\end{tabbing}

\begin{enumerate}

\item	
Implemente como anamorfismo de listas a função
\begin{hscode}\SaveRestoreHook
\column{B}{@{}>{\hspre}l<{\hspost}@{}}%
\column{E}{@{}>{\hspre}l<{\hspost}@{}}%
\>[B]{}\Varid{primes}\mathbin{::}\mathbb{Z}\to [\mskip1.5mu \mathbb{Z}\mskip1.5mu]{}\<[E]%
\ColumnHook
\end{hscode}\resethooks
que deverá, recebendo um número inteiro positivo, devolver a respectiva lista
de factores primos.

A proposta de \ensuremath{\Varid{primes}} deverá vir acaompanhada de um \textbf{diagrama} ilustrativo.

\item A figura mostra a ``\emph{árvore dos primos}'' dos números \ensuremath{[\mskip1.5mu \mathrm{455},\mathrm{669},\mathrm{6645},\mathrm{34},\mathrm{12},\mathrm{2}\mskip1.5mu]}.

      \primes

Com base na alínea anterior, implemente uma função em Haskell que faça a
geração de uma tal árvore a partir de uma lista de inteiros:

\begin{hscode}\SaveRestoreHook
\column{B}{@{}>{\hspre}l<{\hspost}@{}}%
\column{E}{@{}>{\hspre}l<{\hspost}@{}}%
\>[B]{}\Varid{prime\char95 tree}\mathbin{::}[\mskip1.5mu \mathbb{Z}\mskip1.5mu]\to \Conid{Exp}\;\mathbb{Z}\;\mathbb{Z}{}\<[E]%
\ColumnHook
\end{hscode}\resethooks

\textbf{Sugestão}: escreva o mínimo de código possível em \ensuremath{\Varid{prime\char95 tree}} investigando
cuidadosamente que funções disponíveis nas bibliotecas que são dadas podem
ser reutilizadas.%
\footnote{Pense sempre na sua produtividade quando está a programar --- essa
atitude será valorizada por qualquer empregador que vier a ter.}

\end{enumerate}

\Problema

\begin{center}
\fbox{A fornecer na segunda edição deste enunciado}
\end{center}

\Problema

\begin{center}
\fbox{A fornecer na segunda edição deste enunciado}
\end{center}

\newpage
\part*{Anexos}

\appendix

\section{Natureza do trabalho a realizar}
\label{sec:documentacao}
Este trabalho teórico-prático deve ser realizado por grupos de 3 alunos.
Os detalhes da avaliação (datas para submissão do relatório e sua defesa
oral) são os que forem publicados na \cp{página da disciplina} na \emph{internet}.

Recomenda-se uma abordagem participativa dos membros do grupo em \textbf{todos}
os exercícios do trabalho, para assim poderem responder a qualquer questão
colocada na \emph{defesa oral} do relatório.

Para cumprir de forma integrada os objectivos do trabalho vamos recorrer
a uma técnica de programa\-ção dita ``\litp{literária}'' \cite{Kn92}, cujo
princípio base é o seguinte:
%
\begin{quote}\em
	Um programa e a sua documentação devem coincidir.
\end{quote}
%
Por outras palavras, o \textbf{código fonte} e a \textbf{documentação} de um
programa deverão estar no mesmo ficheiro.

O ficheiro \texttt{cp2425t.pdf} que está a ler é já um exemplo de
\litp{programação literária}: foi gerado a partir do texto fonte
\texttt{cp2425t.lhs}\footnote{O sufixo `lhs' quer dizer
\emph{\lhaskell{literate Haskell}}.} que encontrará no \MaterialPedagogico\
desta disciplina des\-com\-pactando o ficheiro \texttt{cp2425t.zip}.

Como se mostra no esquema abaixo, de um único ficheiro (\ensuremath{\Varid{lhs}})
gera-se um PDF ou faz-se a interpretação do código \Haskell\ que ele inclui:

	\esquema

Vê-se assim que, para além do \GHCi, serão necessários os executáveis \PdfLatex\ e
\LhsToTeX. Para facilitar a instalação e evitar problemas de versões e
conflitos com sistemas operativos, é recomendado o uso do \Docker\ tal como
a seguir se descreve.

\section{Docker} \label{sec:docker}

Recomenda-se o uso do \container\ cuja imagem é gerada pelo \Docker\ a partir do ficheiro
\texttt{Dockerfile} que se encontra na diretoria que resulta de descompactar
\texttt{cp2425t.zip}. Este \container\ deverá ser usado na execução
do \GHCi\ e dos comandos relativos ao \Latex. (Ver também a \texttt{Makefile}
que é disponibilizada.)

Após \href{https://docs.docker.com/engine/install/}{instalar o Docker} e
descarregar o referido zip com o código fonte do trabalho,
basta executar os seguintes comandos:
\begin{Verbatim}[fontsize=\small]
    $ docker build -t cp2425t .
    $ docker run -v ${PWD}:/cp2425t -it cp2425t
\end{Verbatim}
\textbf{NB}: O objetivo é que o container\ seja usado \emph{apenas} 
para executar o \GHCi\ e os comandos relativos ao \Latex.
Deste modo, é criado um \textit{volume} (cf.\ a opção \texttt{-v \$\{PWD\}:/cp2425t}) 
que permite que a diretoria em que se encontra na sua máquina local 
e a diretoria \texttt{/cp2425t} no \container\ sejam partilhadas.

Pretende-se então que visualize/edite os ficheiros na sua máquina local e que
os compile no \container, executando:
\begin{Verbatim}[fontsize=\small]
    $ lhs2TeX cp2425t.lhs > cp2425t.tex
    $ pdflatex cp2425t
\end{Verbatim}
\LhsToTeX\ é o pre-processador que faz ``pretty printing'' de código Haskell
em \Latex\ e que faz parte já do \container. Alternativamente, basta executar
\begin{Verbatim}[fontsize=\small]
    $ make
\end{Verbatim}
para obter o mesmo efeito que acima.

Por outro lado, o mesmo ficheiro \texttt{cp2425t.lhs} é executável e contém
o ``kit'' básico, escrito em \Haskell, para realizar o trabalho. Basta executar
\begin{Verbatim}[fontsize=\small]
    $ ghci cp2425t.lhs
\end{Verbatim}

\noindent Abra o ficheiro \texttt{cp2425t.lhs} no seu editor de texto preferido
e verifique que assim é: todo o texto que se encontra dentro do ambiente
\begin{quote}\small\tt
\text{\ttfamily \char92{}begin\char123{}code\char125{}}
\\ ... \\
\text{\ttfamily \char92{}end\char123{}code\char125{}}
\end{quote}
é seleccionado pelo \GHCi\ para ser executado.

\section{Em que consiste o TP}

Em que consiste, então, o \emph{relatório} a que se referiu acima?
É a edição do texto que está a ser lido, preenchendo o anexo \ref{sec:resolucao}
com as respostas. O relatório deverá conter ainda a identificação dos membros
do grupo de trabalho, no local respectivo da folha de rosto.

Para gerar o PDF integral do relatório deve-se ainda correr os comando seguintes,
que actualizam a bibliografia (com \Bibtex) e o índice remissivo (com \Makeindex),
\begin{Verbatim}[fontsize=\small]
    $ bibtex cp2425t.aux
    $ makeindex cp2425t.idx
\end{Verbatim}
e recompilar o texto como acima se indicou. (Como já se disse, pode fazê-lo
correndo simplesmente \texttt{make} no \container.)

No anexo \ref{sec:codigo} disponibiliza-se algum código \Haskell\ relativo
aos problemas que são colocados. Esse anexo deverá ser consultado e analisado
à medida que isso for necessário.

Deve ser feito uso da \litp{programação literária} para documentar bem o código que se
desenvolver, em particular fazendo diagramas explicativos do que foi feito e
tal como se explica no anexo \ref{sec:diagramas} que se segue.

\section{Como exprimir cálculos e diagramas em LaTeX/lhs2TeX} \label{sec:diagramas}

Como primeiro exemplo, estudar o texto fonte (\lhstotex{lhs}) do que está a ler\footnote{
Procure e.g.\ por \texttt{"sec:diagramas"}.} onde se obtém o efeito seguinte:\footnote{Exemplos
tirados de \cite{Ol05}.}
\begin{eqnarray*}
\start
\ensuremath{\Varid{id}\mathrel{=}\conj{\Varid{f}}{\Varid{g}}}
\just\equiv{ universal property }
\ensuremath{\begin{lcbr}\p1\comp \Varid{id}\mathrel{=}\Varid{f}\\\p2\comp \Varid{id}\mathrel{=}\Varid{g}\end{lcbr}}
\just\equiv{ identity }
\ensuremath{\begin{lcbr}\p1\mathrel{=}\Varid{f}\\\p2\mathrel{=}\Varid{g}\end{lcbr}}
\qed
\end{eqnarray*}

Os diagramas podem ser produzidos recorrendo à \emph{package} \Xymatrix, por exemplo:
\begin{eqnarray*}
\xymatrix@C=2cm{
    \ensuremath{\N_0}
           \ar[d]_-{\ensuremath{\cataNat{\Varid{g}}}}
&
    \ensuremath{\mathrm{1}\mathbin{+}\N_0}
           \ar[d]^{\ensuremath{\Varid{id}\mathbin{+}\cataNat{\Varid{g}}}}
           \ar[l]_-{\ensuremath{\mathsf{in}}}
\\
     \ensuremath{\Conid{B}}
&
     \ensuremath{\mathrm{1}\mathbin{+}\Conid{B}}
           \ar[l]^-{\ensuremath{\Varid{g}}}
}
\end{eqnarray*}

\section{Código fornecido}\label{sec:codigo}

\subsection*{Problema 1}

\begin{hscode}\SaveRestoreHook
\column{B}{@{}>{\hspre}l<{\hspost}@{}}%
\column{E}{@{}>{\hspre}l<{\hspost}@{}}%
\>[B]{}\Varid{h}\mathbin{::}[\mskip1.5mu \Conid{Int}\mskip1.5mu]{}\<[E]%
\ColumnHook
\end{hscode}\resethooks

%----------------- Soluções dos alunos -----------------------------------------%

\section{Soluções dos alunos}\label{sec:resolucao}
Os alunos devem colocar neste anexo as suas soluções para os exercícios
propostos, de acordo com o ``layout'' que se fornece.
Não podem ser alterados os nomes ou tipos das funções dadas, mas pode ser
adicionado texto ao anexo, bem como diagramas e/ou outras funções auxiliares
que sejam necessárias.

\noindent
\textbf{Importante}: Não pode ser alterado o texto deste ficheiro fora deste anexo.

\subsection*{Problema 1}

\begin{hscode}\SaveRestoreHook
\column{B}{@{}>{\hspre}l<{\hspost}@{}}%
\column{E}{@{}>{\hspre}l<{\hspost}@{}}%
\>[B]{}\Varid{hindex}\mathrel{=}\bot {}\<[E]%
\ColumnHook
\end{hscode}\resethooks

\subsection*{Problema 2}
Primeira parte:
\begin{hscode}\SaveRestoreHook
\column{B}{@{}>{\hspre}l<{\hspost}@{}}%
\column{E}{@{}>{\hspre}l<{\hspost}@{}}%
\>[B]{}\Varid{primes}\mathrel{=}\bot {}\<[E]%
\ColumnHook
\end{hscode}\resethooks
Segunda parte:
\begin{hscode}\SaveRestoreHook
\column{B}{@{}>{\hspre}l<{\hspost}@{}}%
\column{E}{@{}>{\hspre}l<{\hspost}@{}}%
\>[B]{}\Varid{prime\char95 tree}\mathrel{=}\bot {}\<[E]%
\ColumnHook
\end{hscode}\resethooks

%----------------- Índice remissivo (exige makeindex) -------------------------%

\printindex

%----------------- Bibliografia (exige bibtex) --------------------------------%

\bibliographystyle{plain}
\bibliography{cp2425t}

%----------------- Fim do documento -------------------------------------------%
\end{document}
